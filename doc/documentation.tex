\documentclass[a4paper,12pt]{article}
\usepackage[utf8x]{inputenc}
\usepackage{polski}
%\usepackage[vmargin=2cm]{geometry}
\usepackage{amsmath}
\usepackage{textcomp}
\usepackage{hyperref}

%opening
\title{Diaballik - dokumentacja projektu}
\author{Tomasz Zakrzewski, numer indeksu: 336079}

% Margins
\topmargin=-0.45in
\evensidemargin=0in
\oddsidemargin=0in
\textwidth=6.5in
\textheight=9.0in
\headsep=0.25in

\linespread{1.1} % Line spacing

\begin{document}

\maketitle

\setcounter{tocdepth}{2}

\tableofcontents

\setcounter{section}{0}
\pagebreak

\section{Wstęp}
Niniejsza dokumentacja podzielona jest na kilka głównych części - krótki wstęp (obecnie czytany fragment), krótki opis planowanej funkcjonalności 
(dosyć mocno pokrywający się z opisem z moodle'a + kilka moich pomysłów) i założenia projektowe, w których niejako tłumaczę się z kolejnej sekcji -
opisu klas i podziału odpowiedzialności. Na deser kilka mniej ważnych technikaliów jak format zapisu plików ze stanem gry oraz prawdopodobne pliki
konfiguracyjne.

\section{Planowana funkcjonalność}
Moodle:
\begin{itemize}
\item program umożliwiający grę w Diaballika w opcjach człowiek vs człowiek, komputer vs człowiek i komputer vs komputer
\item wygodne UI umożliwiające sensowne podawanie ruchów i dosyć wygodną oraz intuicyjną obsługę
\item ``nietrywialna'' inteligencja komputera - główny moduł SI oparty o MCTS
\item możliwość uzyskania podpowiedzi od komputera
\item swobone cofanie i przewijanie do przodu po historii ruchów
\item edytor stanu gry pozwalający kompletnie (jednak zgodnie z zasadami) zmienić stan gry włącznie z aktywnym graczem
\item możliwość zapisu/wczytania gry
\end{itemize}
Pomysły własne:
\begin{itemize}
\item zamiast posiadać na trwałe wrzuconą w projekt sztuczną inteligencję, chcę aby mój program pozwalał na podpinanie dowolnego bota napisanego
zgodnie z rozszerzonym protokołem GTP (z wydziałowego konkursu na bota), który skrótowo opiszę w sekcji 3.
\item chciałbym (prawodpodobnie dopiero podczas jakichś własnych prób rozwijania projektu  w wolnym czasie, po terminie oddania) udostępnić możliwość
gry po sieci. Nie planuję tej funkcjonalności w wersji, którą oddam na czerwcowy termin, ale nadmieniam to aby usprawiedliwić pewne dalsze decyzje 
projektowe.
\item z podobnego powodu i w podobnym kontekście podaję intencję posiadania zamiast pionków jakichś bardziej skomplikowanych animacji ``ludzików'' na
planszy oraz podań piłki między ludzikami
\end{itemize}

\section{Założenia}
Brałem udział w wydziałowym konkursie na SI do Diaballika. Wiedząc jednak że nie mogę mieć pewności, iż na owym konkursie ugram sobie ocenę/bonusa,
chciałem bota napisać w jak najbardziej przenoszalny sposób. Stąd pomysł na podział całości na 3 niezależne projekty: DiaballikEngine, DiaballikBot
oraz Diaballik. 

Kolejno są to: silnik do gry - obsługuje całą logikę gry, zawiera wszystkie klasy potrzebne do przeprowadzenia rozgrywki całkowicie
niezależnej od UI - z punktu widzenia silnika może on być nawet używany jako serwer do gier i nic nie stoi temu na przeszkodzie. 

DiaballikBot, to jak nazwa wskazuje, bot do grania w Diaballika. Mam przygotowane 2 wersje bazujące na silniku (jako 2 gałęzi drzewa git, obie 
korzystają z silnika jako sumbodułu). Pierwsza wersja (v0.1 aka NullAI) była de facto testem sprawności silnika grającym kompletnie losowo. 
Druga wersja (obecnie v0.2.4 aka MCTSAI) to sztuczna inteligencja oparta o MCTS z losowymi rozgrywkami z pewnymi heurystycznymi usprawnieniami 
wydajności oraz jakości partii. Obie wersje będą dostępne jako przeciwnicy w grze.

Diaballik to gra z GUI wymagana jako końcowy projekt. Będzie korzystała z silnika do realizacji mechaniki gry, z Qt do GUI (i nie tylko) oraz będzie
obsługiwać boty (i prawdopodobnie posiadać wbudowaną okrojoną wersję jakiegoś bota do podawania hintów).

\section{Klasy i podział obowiązków}
\subsection{DiaballikEngine}
Projekt zawiera następujące klasy z następującym podziałem obowiązków:
\begin{itemize}
 \item \verb|Point| - prosta klasa na punkt 2D, przydatny w wielu przypadkach,
 \item \verb|Move| - prosta klasa, właściwie mogłaby być zastąpiona std::pair. Jako para punktów pełni rolę wygodnej struktury do przekazywania informacji
 o ruchu - skąd dokąd.
 \item \verb|BitContainer| - zaimplementowana na \verb|vector<uint8_t>| klasa przechowująca inty w skompresowanej postaci i dająca proste metody dostępu do nich.
 Używana do efektywnego pamięciowo przechowywania planszy.
 \item \verb|Board| - klasa służąca jako opakowanie wokół \verb|BitContainera| - logicznie reprezentuje planszę i udostępnia wiele operacji na planszy - 
 przesuwanie pionka po współrzędnych, szukanie różnego rodzaju własności ustawień na planszy itd. \verb|BitContainer| działa jak tablica jednowymiarowa, więc
 Board tłumaczy punkty na jednowymiarowe współrzędne i odwrotnie.
 \item \verb|Game| - nadrzędna klasa, której programista powinien końcowo używać. Zawiera obiekt \verb|Board| do przechowywania planszy, podczas 
 wykonywania operacji na planszy sprawdza ich zgodność z zasadami gry, ustala zwycięzców, większość możliwych do wykrycia remisów itd. Posiada wiele funkcji
 zwracających informacje o ustawieniach pionków na planszy.
\end{itemize}
Cały projekt zaimplementowany jest z myślą o przenośności i optymalnym zarządzaniu głównie pamięcią, zaraz potem czasem. Podyktowane to było jego
planowanym użyciem do pisania bota, mającego działać w mocno kontrolowanym środowisku ze skończonymi zasobami.
Pionki na planszy oraz gracze i typy ruchów są reprezentowane przez enumy (kolejno: \verb|FieldType|, \verb|GamePlayer|, \verb|MoveType|).
Projekt posiada własny system budowania na \verb|CMake| i jest kompilowany do statycznej biblioteki.

\subsection{DiaballikBot, GTP-E}
Bot był zbudowany na silniku, więc dużo nie trzeba było dopisywać:
\begin{itemize}
 \item \verb|Singleton<T>| - generyczna klasa \verb|Singletonu| parametryzowana klasą, która w efekcie ma być \verb|Singletonem|. Po tej klasie 
 dziedziczą wszystkie \verb|Singletony|.
 \item \verb|CommunicationHandler| - Singleton opakowywujący standardowe wejście i udostępniajacy kilka dodatkowych prostych funkcji związanych z 
 wejściem/wyjściem
 \item \verb|GTPParser| - \verb|Singleton| służący do parsowania komunikatów do SI w protkole GTP. Uwaga, docelowo boty obsługiwane przez grę 
 będą musiały obsługiwać rozszerzony protokół, co zostanie w parserze uwzględnione. Parser pamięta wskaźnik na obiekt \verb|AI|.
 \item \verb|AI| - abstrakcyjna klasa zapewniająca częściową implementację metod \verb|play| i \verb|gen_move| odpowiadających takim samym 
 poleceniom w GTP. Implementacja zawarta w \verb|AI| zajmuje się tylko rejestrowaniem ruchów w chronionym atrybucie game (obiekcie klasy 
 \verb|Game| z silnika) oraz rozróżnianiem, którym graczem jest \verb|AI| (białym/czarnym) na podstawie pierwszego ruchu.
 \item \verb|MCTSAI|, \verb|NullAI| - klasy dziedziczące po \verb|AI|. Każda klasa implementuje swój własny model sztucznej inteligencji. 
 Widać, że bardzo łatwo można dopisać nowe strategie tworząc jedynie klasę dziedziczącą po \verb|AI| i przedefiniowywując część metod.
\end{itemize}
Aby móc obsługiwać cofanie gry oraz edycję planszy, boty w komunikacji muszą implementować rozszerzoną wersję protokołu \verb|GTP| z turnieju. 
Idea jest taka, że każdy bot miałby plik konfiguracyjny w swoim katalogu czytany przez grę, zawierający m.in. flagę informującą czy bot umie 
używać rozszerzonego protokołu. Jeśli tak - gra pozwalałaby cofać ruchy i edytować planszę. Jeśli nie, dostępna byłaby tylko standardowa gra.

Samo rozszerzenie protokołu zawiera 2 nowe polecenia:
\begin{verbatim}undo_move <g>\end{verbatim}
Wysyłane do obydwu graczy, mówiące że ostatni ruch gracza o kolorze pionka \verb|<g>| (opis zgodny ze stroną turnieju - \verb|<g>| to opis koloru
gracza, w praktyce 'b' lub 'w' od 'black' oraz 'white'), ma zostać cofnięty.
\begin{verbatim}new_game <p> <b> <g>\end{verbatim}
Gdzie \verb|<p>| to opis pól zajętych przez pionki kolejno gracza czarnego i białego podany w formacie zgodnym z \verb|<r>| z oryginalnego 
protokołu (ciąg nieoddzielonych białymi znakami wspórzędnych pól postaci \verb|[a-g][1-7]|). W tym samym formacie jest \verb|<b>| - wskazuje
zawsze 2 pola na których jest kolejno piłka gracza czarnego i białego. \verb|<g>| informuje, który gracz ma teraz wykonać ruch. Np. ustawienie
takie jak na początku gry (gracze ustawieni w liniach po przeciwnych stronach planszy, czarny na górze, zaczyna biały) byłoby wywoływane 
poleceniem:
\begin{verbatim}new_game a7b7c7e7f7g7a1b1c1e1f1g1 d7d1 w\end{verbatim}
Zmieniony protokół otrzymał autorską, roboczą nazwę \verb|GTP-E| (GTP-Extended).

\subsection{Diaballik}
W tej, najważniejszej w tym dokumencie, części projektu chcę trzymać się bardzo ważnego moim zdaniem założenia o rozdzieleniu Qt od reszty aplikacji.
Z tego powodu wprowadzam dosyć wyraźny podział między klasami korzystającymi/dziedziczącymi po czymś z Qt i tymi całkiem niezależnymi. Podział ten
naturalnie wyznacza mi biblioteka DiaballikEngine - używam jej jako silnika gry kompletnie niezwiązanego z warstwą GUI. Pozostałe klasy:
\begin{itemize}
 \item \verb|MainWindow| - dziedziczące po \verb|QMainWindow| główne okno aplikacji. Zajmuje się konstrukcją siebie i widocznych komponentów, 
 podpięciem najważniejszych sygnałów itp. Wszystko co związane z GUI i częściowo z jego zachowaniem. Widgetem służącym do wyświetlania gry będzie
 \verb|QGraphicsView|.
 \item \verb|GraphicsScene| - klasa dziedzicząca po \verb|QGraphicsScene| z przedefiniowanymi metodami do obsługi inputu. Będzie podawana do 
 \verb|QGraphicsView| jako aktywna scena. 
 \item \verb|GraphicsBackgroundTile| - klasa dziedziczącą po jednym z \verb|QGraphicsItem| (prawdopodobnie \verb|QGraphicsPixmapItem|, ale 
 poważnie zastanawiam się nad \verb|QGraphicsSVGItem|). Reprezentuje pojedynczy klocek z tła planszy (klocków jest razem 49). Istnienie tej klasy
 argumentuję wygodną możliwością wyłapywania i obsługiwania kliknięć na puste pola (np. rysowania jakichś rzeczy na planszy, podświetlenia pól
 itd.).
 \item \verb|GraphicsPlayerTile| - analogiczna do \verb|GraphicsBackgroundTile| klasa reprezentująca pionek gracza (z piłką lub bez).
 \item \verb|GameHandler| (\verb|Singleton| i/lub \verb|QObject|) - serce aplikacji. Jej odpowiedzialnością będzie zajmowanie się całą logiką
 aplikacji (nie mylić z logiką gry!). Będzie zawierać jako prywatny atrybut obiekt klasy \verb|Game|, na podstawie którego będzie w stanie 
 walidować i wykonywać ruchy otrzymywane od graczy oraz ustawiać nową grę i realizować od strony logiki wszystkie funkcjonalności założone
 w projekcie.
 \item \verb|Player| - abstrakcyjna klasa reprezentująca gracza. Jako, że potrzebne będzie wiele rodzajów graczy, komunikujących się z 
 aplikacją na zupełnie różne sposoby, \verb|GameHandler| będzie posiadać 2 wskaźniki typu \verb|Player|, pod które mogą być podpięte dowolne
 obiekty klas dziedziczących po oraz implementujących odpowiednie metody z klasy \verb|Player|. Na chwilę obecną, zarys jest następujący: klasa
 \verb|Player| ma zadeklarowane (\verb|virtual|) metody \verb|bool isMoveReady()| oraz \verb|Move getMove()| i \verb|void setMove()|. 
 Planuję korzystając z mechanizmu wątków uruchamiać playerów w osobnych wątkach i pozwalać im na swoje sposoby realizować komunikację 
 (np. \verb|HumanPlayer| miałby sloty do których podłączone byłyby sygnały od \verb|GraphicsScene| oraz itemów, klasa odpowiedzialna za sztuczną
 inteligencję miałaby metody służące do komunikacji przez \verb|stdin/stdout| z botem itd.). Playerzy byliby w \verb|QTimerze| odpytywani o 
 istnienie nowych ruchów i ruchy te przekazywane byłyby do \verb|GameHandlera|, odpowiedzialnego za ich wykonanie.
 \item \verb|HumanPlayer| - wcześniej wspomniana planowana dziedzicząca po \verb|Player| klasa reprezentująca człowieka grającego w grę. 
 Dotychczasowym pomysłem jest, jako że gracz posiada stałą liczbę pionków, będących przez całą grę tymi samymi obiektami klasy 
 \verb|GraphicsPlayerTile|, aby podpinać eventy kliknięć na pionki gracza pod odpowiedni obiekt \verb|HumanPlayer| reprezentujący gracza 
 (na starcie gry). Jeśli chodzi o pola puste to najlepszą w tej chwili koncepcją jest podpiąć je pod \verb|GameHandlera| i pozwolić mu na 
 stwierdzenie do którego gracza odnosi się kliknięcie na pole LUB zaimplementowanie \verb|GraphicsBackgroundTile| tak, aby odpytywał o tę
 informację \verb|GameHandlera| i sam przekazywał ją \verb|HumanPlayerom|. 
 \item \verb|AIPlayer| - klasa dziedzicząca po \verb|Player| odpowiedzialna za komunikację z botem po \verb|stdin/stdout|. Reprezentuje komputer
 grający jakimś algorytmem sztucznej inteligencji.
 \item(Bonus, do zrealizowania pewnie po terminie projektu) \verb|NetworkPlayer| - gracz sieciowy zaimplementowany za pomocą socketów 
 \verb|TCP/IP|.
 \item możliwe jest jeszcze dodanie klasy \verb|AIHintPlayer|, lub czegoś podobnego, służącej do generowania podpowiedzi dla gracza - 
 nie udało mi się jednoznacznie rozstrzygnąć czy taka klasa będzie potrzebna. Jednak, jeśli osobna klasa do tego miałaby powstać - dziedziczyłaby
 po \verb|Playerze| i komunikacja z nią odbywałaby się podobnie.
\end{itemize}
Jak widać, jedynym połączeniem między warstwą aplikacji/GUI a czystą logiką gry jest istnienie atrybutu typu \verb|Game| w \verb|GameHandlerze|. 
Jako że Game dostarcza pewną wartwę abstrakcji od leżących niżej mocno technicznych rozwiązań związanych z planszą, grą itd., widać że osiągnięty 
będzie dosyć sensowny rozdział poszczególnych fragmentów programu, co było jednym z głównych założeń po doświadczeniach autora z ``Przesuwanką''.

Z rzeczy niewymienionych wyżej, w aplikacji, do obsługi stanów mam zamiar skorzystać z mechanizmu grafu stanów z Qt (
\href{http://qt-project.org/doc/qt-4.8/qstate.html}{QState}, \href{http://qt-project.org/doc/qt-4.8/qstatemachine.html}{QStateMachine}). 
\section{Zapis gry i ustawienia}
Wszystkie dane gry tworzone na maszynie, na której aplikacja jest uruchamiana, będą trzymane w $\sim$\verb|/.diaballik|. W tej chwili w tym
katalogu planowane są: plik \verb|config.ini| z ustawieniami aplikacji (wszystko co przyjdzie do głowy programiście podczas pisania, co możnaby
regulować ustawieniami), folder \verb|bots|, zawierający osobne foldery dla każdego bota (zawierające binarkę i wspomniany plik konfiguracyjny, 
ale też dowolne pliki wykorzystywane przez bota) oraz folder \verb|saves| zawierający pliki \verb|*.sav| - pliki zapisu stanu gry.

Format poprawnego pliku \verb|*.sav| jest następujący: 
\begin{verbatim}
 DIABSAVE<p><b><g><h>
\end{verbatim}
Gdzie: \verb|<p><b><g>| to kolejno informacje takie jak w komendzie new\_game z GTP-E, tylko zapisane binarnie i bez spacji; współrzędne przetłumaczone
z \verb|[a-g][1-7]| do \verb|[0-6][0-6]|, 3 bity na każdą wspołrzędną. \verb|<g>| zapisywane bedzie za pomocą 1 bitu - 0 oznaczać będzie gracza
białego, a 1 czarnego.

\verb|<h>| to historia ruchów. Jako, że ruch jest sciśle zależny od gracza, a aktualnego gracza znamy już poprzednich danych, informacji o aktywnym 
w danym momencie gry graczu nie będzie w historii ruchów. Ruch jest parą punktów - skąd dokąd (rozróżnieniem typu ruchu zajmuje się logika gry).
Punkty będą zapisywane w taki sam sposób jak powyżej (binarnie) bez odstępów. Aby oddzielić koniec kolejki jednego gracza od początku kolejki 
kolejnego, wstawiane będą 3 zapalone bity, jako że wartość 7 nie jest poprawną współrzędną. Ruchy będą w kolejności od początku gry do stanu 
obecnego.

Przy przyjętym tutaj modelu nic nie stoi na przeszkodzie, żeby zapisywać niedokończony (niezaakceptowany jeszcze przez gracza) ruch, jako że można
założyć wymaganie iż ruch jest skończony, jeśli stoi za nim sekwencja 3 zapalonych bitów (co z przyjętych założeń i tak jest spełnione dla wszystkich
ruchów poza ostatnim więc dodanie 3 bitów na końcu pliku jest bardzo małym kosztem za dużą dodatkową funkcjonalność).

\section{Odnośniki}
Wszystkie 3 projekty w aktualnej wersji można obejrzeć (i forkować, komentować itd.) na moich repozytoriach na githubie:\linebreak
\url{https://github.com/Tommalla/DiaballikEngine}\linebreak 
\url{https://github.com/Tommalla/DiaballikBot}\linebreak
\url{https://github.com/Tommalla/Diaballik}

\end{document}
