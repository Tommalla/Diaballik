\documentclass[a4paper,12pt]{article}
\usepackage[utf8x]{inputenc}
\usepackage{polski}
\usepackage[vmargin=2cm]{geometry}
\usepackage{amsmath}

%opening
\title{Diaballik - dokumentacja projektu}
\author{Tomasz Zakrzewski, numer indeksu: 336079}

\begin{document}

\maketitle

\begin{abstract}
\section{Wstęp}
Skrótowo o tym dokumencie
\section{Planowana funkcjonalność}
Przeklejka z módla + to co bym chciał potem
\section{Założenia}
Podział na 3 projekty
\section{Klasy i odpowiedzialności}
\subsection{DiaballikEngine}
\subsection{DiaballikBot, GTP-E}
\subsubsection{NullAI}
\subsubsection{MCTSAI}
\subsection{Diaballik}
\section{Zapis gry i ustawienia}
\end{abstract}

\setcounter{section}{0}

\section{Wstęp}
Niniejsza dokumentacja podzielona jest na kilka głównych części - krótki wstęp (obecnie czytany fragment), krótki opis planowanej funkcjonalności 
(dosyć mocno pokrywający się z opisem z moodle'a + kilka moich pomysłów) i założenia projektowe, w których niejako tłumaczę się z kolejnej sekcji -
opisu klas i podziału odpowiedzialności. Na deser kilka mniej ważnych technikaliów jak format zapisu plików ze stanem gry oraz prawdopodobne pliki
konfiguracyjne

\section{Planowana funkcjonalność}
Moodle:
\begin{itemize}
\item program umożliwiający grę w Diaballika w opcjach człowiek vs człowiek, komputer vs człowiek i komputer vs komputer
\item wygodne ui umożliwiające sensowne podawanie ruchów i dosyć wygodną oraz intuicyjną obsługę
\item ``nietrywialna'' inteligencja komputera - główny moduł SI oparty o MCTS
\item możliwość uzyskania podpowiedzi od komputera
\item swobone cofanie i przewijanie do przodu po historii ruchów
\item edytor stanu gry pozwalający kompletnie (zgodnie z zasadami) zmienić stan gry włącznie z aktywnym graczem
\item możliwość zapisu/wczytania gry
\end{itemize}
Pomysły własne:
\begin{itemize}
\item zamiast posiadać na trwałe wrzuconą w projekt sztuczną inteligencję, chcę aby mój program pozwalał na podpinanie dowolnego bota napisanego
zgodnie z rozszerzonym protokołem GTP (z wydziałowego konkursu na bota), który skrótowo opiszę w sekcji 3.
\item chciałbym (prawodpodobnie dopiero podczas jakichś własnych prób rozwijania projektu po terminie oddania w wolnym czasie) udostępnić możliwość
gry po sieci. Nie planuję tej funkcjonalności w wersji, którą oddam na czerwcowy termin, ale nadmieniam to aby usprawiedliwić pewne dalsze decyzje 
projektowe.
\item z podobnego powodu i w podobnym kontekście podaję intencję posiadania zamiast pionków jakichś bardziej skomplikowanych animacji ``ludzików'' na
planszy oraz podań piłki między ludzikami
\end{itemize}

\section{Założenia}
Brałem udział w wydziałowym konkursie na SI do Diaballika. Wiedząc jednak że nie mogę mieć pewności, iż na owym konkursie ugram sobie ocenę/bonusa,
chciałem bota napisać w jak najbardziej przenoszalny sposób. Stąd pomysł na podział całości na 3 niezależne projekty: DiaballikEngine, DiaballikBot
oraz Diaballik. Kolejno są to: silnik do gry - obsługuje całą logikę gry, zawiera wszystkie klasy potrzebne do przeprowadzenia rozgrywki całkowicie
niezależnej od UI - z punktu widzenia silnika może on być nawet używany jako serwer do gier i nic nie stoi temu na przeszkodzie. DiaballikBot, to jak
nazwa wskazuje, bot do grania w Diaballika. Mam przygotowane 2 wersje bazujące na silniku (jako 2 gałęzi drzewa git, obie korzystają z silnika jako
sumbodułu).
\end{document}
